\documentclass{article}

\usepackage{ctex}

\begin{document}
	2020-8-19\qquad Yanjie Ze\\
	习题4.1 用极大似然估计法推出朴素贝叶斯法中的概率估计公式(4.8)及公式(4.9)\\
	证明:
	$$设\  P(Y=c_k)=\theta$$
	
	$$令\ \displaystyle m = \sum_{i=1}^{N}I(y_i=c_k)$$
	
	$$似然函数为\ L(\theta)={N \choose m}\theta^m(1-\theta)^{N-m}$$
	
	$$\Rightarrow \ln{L(\theta)}=\ln{N \choose m} + m\ln\theta + (N-m)\ln{(1-\theta)}  $$
	
	$$\Rightarrow {\partial\ln{L(\theta)}\over \partial \theta}={m \over \theta }+{N-m\over \theta-1 }=0$$
	
	$$\Rightarrow \theta = {m \over N}$$
	
	$$\Rightarrow P(Y=c_k)=\theta={m \over N}={\sum_{i=1}^{N}I(y_i=c_k)\over N}$$
	
	公式(4.8) 得证
	
	
	
	$$设\ P(X^{(j)}=a_{jl}\vert Y=c_k)=\beta$$
	
	$$令\ \displaystyle m=\sum_{i=1}^NI(y_i=c_k),q=\sum_{i=1}^NI(x_i^{(j)}=a_{jl},y_i=c_k)$$
	
	$$似然函数为\  L(\beta)={m \choose q}\beta^q(1-\beta)^{m-q} $$
	
	$$\Rightarrow \ln{L(\beta)}= \ln{m \choose q}+q\ln\beta+(m-q)\ln(1-\beta)$$
	
	$$\Rightarrow {\partial \ln{L(\beta)}\over \partial \beta }= {q \over \beta}+ {m-q \over \beta -1}=0$$
	
	$$\Rightarrow \beta={q\over m}$$
	
	$$\Rightarrow P(X^{(j)}=a_{jl}\vert Y=c_k)=\beta={q\over m}={\sum_{i=1}^NI(x_i^{(j)}=a_{jl},y_i=c_k)\over \sum_{i=1}^NI(y_i=c_k)}$$

	公式(4.9) 得证
\end{document}